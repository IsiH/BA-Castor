\section*{Abstract}

In current particle physics research the search for new particles is a major motivation for most experiments. To be able to correctly find unknown events a full event classification is the best available method. In this thesis a full event classification with the CASTOR calorimeter as part of the CMS detector at CERN is evaluated. The classification is made with the help of convolutional neural networks. Simulated events generated by the Monte Carlo generators PYTHIA8 and EPOS are used as a reference. To simplify the analysis the events are split into only two categories, electromagnetic events containing electrons and photons with their respective antiparticles, and hadronic events. Furthermore only single particles with no other particle crossing CASTOR in the same event, or isolated particles with a distance of at least $\Delta$R = 0.8 to the next particle in the same event are included in the training data. The trained network is evaluated on real data measured by CASTOR. It turned out that real data contains only a small fraction of single or isolated particles which lead to the fact that the neural network classification could not perform well to separate these categories. It is found that if the predicted possibility of one event belonging to one category was 80 \% or higher, events are correctly identified. Thus, with more refined networks or preprocessed data, such an event classification could possibly reach human performance.  

\vspace{2cm}

In der aktuellen Teilchenphysikforschung ist die Suche nach neuen Teilchen eines der größten Aufgabenfelder. Um eine korrekte Trennung zwischen bekannten und unbekannten Events zu ermöglichen, muss eine vollständige Eventklassifizierung möglich sein. In dieser Arbeit ist eine solche Klassifizierung am CASTOR-Kalorimeter als Teil des CMS Detektors am CERN durchgeführt worden. Dafür wurde ein neuronales Netz mit Convolutional Layers verwendet, welches die Daten als Bilder interpretierte. Für das Training des Netzwerkes wurden von den Monte Carlo Generatoren PYTHIA8 und EPOS generierte Events verwendet. Um die Analyse zu vereinfachen, sind die Events in zwei Kategorien aufgeteilt, elektromagnetische Events mit Elektronen und Photonen und deren Antiteilchen und hadronische Events. Zudem werden nur isolierte Teilchen in die Trainingsdaten aufgenommen, welche entweder einen Mindestabstand von $\Delta$R = 0.8 zu ihrem nächsten Nachbarn hatten oder ohne weiteres Teilchen innerhalb von CASTOR detektiert wurden. Das trainierte Netzwerk wird dananch an echten Daten getestet. Da diese nur zu einem geringen Anteil isolierte Events enthalten, kann das Netzwerk diese Daten nicht vollständig den einzelnen Kategorien zuordnen. Nur wenn die vorhergesagte Wahrscheinlichkeit bei über 80 \% liegt, sind ausgewählte Events richtig klassifiziert. Mithilfe von besser zugeschnittenen neuronalen Netzen oder vorverarbeiteten Daten kann das Netzwerk verbessert werden, um möglicherweise eine Eventklassifizierung zu vereinfachen.
\cleardoublepage
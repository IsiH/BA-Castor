\section*{Abstract}

In current particle physics research the search for new particles is a major part in most experiments. To be able to correctly separate unknown from known events a full event classification is necessary. In this thesis a full event classification with the CASTOR calorimeter as part of the CMS detector at CERN is done. The classification is made with the help of convolutional neural networks, treating the input data as images. Simulated events generated by the Monte Carlo generators PYTHIA8 and EPOS are used. To simplify the analysis the events are split into only two categories, electromagnetic events containing electrons and photons with their respective antiparticles and hadronic events. Furthermore only single particles with no other particle crossing CASTOR during this event or isolated particles with a distance of at least $\Delta$R = 0.8 to the next particle are included in the training data. The trained network is evaluated on real data measured by CASTOR. As real data contains only a small amount of single or isolated events the classification could not correctly separate the categories. If the predicted possibility of one event belonging to one category was 80 \% or higher, sampled events are correctly identified. With a better performance, reachable by more refined networks or preprocessed data, such an event classification could possibly reach human performance.  

\vspace{2cm}

In der aktuellen Teilchenphysikforschung ist die Suche nach neuen Teilchen eines der größten Aufgabenfelder. Um eine korrekte Trennung zwischen bekannten und unbekannten Events zu ermöglichen, muss eine vollständige Eventklassifizierung möglich sein. In dieser Arbeit ist eine solche Klassifizierung am CASTOR-Kalorimeter als Teil des CMS Detektors am CERN durchgeführt worden. Dafür wurde ein neuronales Netz mit Convolutional Layers verwendet, welches die Daten als Bilder interpretierte. Für das Training des Netzwerkes wurden von den Monte Carlo Generatoren PYTHIA8 und EPOS generierte Events verwendet. Um die Analyse zu vereinfachen, sind die Events in zwei Kategorien aufgeteilt, elektromagnetische Events mit Elektronen und Photonen und deren Antiteilchen und hadronische Events. Zudem werden nur isolierte Teilchen in die Trainingsdaten aufgenommen, welche entweder einen Mindestabstand von $\Delta$R = 0.8 zu ihrem nächsten Nachbarn hatten oder ohne weiteres Teilchen innerhalb von CASTOR detektiert wurden. Das trainierte Netzwerk wird dananch an echten Daten getestet. Da diese nur zu einem geringen Anteil isolierte Events enthalten, kann das Netzwerk diese Daten nicht vollständig den einzelnen Kategorien zuordnen. Nur wenn die vorhergesagte Wahrscheinlichkeit bei über 80 \% liegt, sind ausgewählte Events richtig klassifiziert. Mithilfe von besser zugeschnittenen neuronalen Netzen oder vorverarbeiteten Daten kann das Netzwerk verbessert werden, um möglicherweise eine Eventklassifizierung zu vereinfachen.
\cleardoublepage
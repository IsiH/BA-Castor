CASTOR is a electromagnetic and hadronic calorimeter, which measures the very forward region of CMS. In future it can be used to detect and identify new particles such as strangelets. For this a full event classification of known particles has been tried using neural networks. In this thesis only detector data has been used without reconstruction of particle data. To simplify the analysis, only particle types were tried to separate. Three classes were implemented, electromagnetic particles, which included photons and electrons, hadronic particles, mostly pions, and background which was defined differently for different data sets. The implementation of the network was done with keras, using tensorflow, a python based machine learning framework, as a backend. As calorimeter data represented by a histogram of the energy deposit per section can be seen as an image, classical image recognition methods were applied. This meant a network consisting mostly of convolutional layers, which can detect spatially local features better and faster than conventional neural networks. \newline
The network was trained on three different data sets, isolated particles, single particles and a binary classification, omitting the background category. Several network designs have been tested, finally deciding on a network with 6 hidden layers. Deeper networks have not shown a better performance. The network performance was continuously tested on validation data to oversee the learning progress and to possibly prevent overfitting. In the end the network did not perform very well on either data set. Used on real data it was able to identify events correctly, which had a very high probability of belonging to one class. Those events which bordered on 50 \% probability were often wrong and contributed to the poor performance of the network. This may be due to the fact that physical information was not conveyed through the classic convolutional layers. The particles detected by CASTOR always enter the calorimeter from the same direction. Therefore more emphasis should be placed on the left sections of every tower in further analyses. \newline
In conclusion, the application of neural networks in particle physics is vast and has been shown in several other experiments. For a full event classification, with the limited methods used in this thesis, using only calorimeter data is a major challenge for the network. A significant preprocessing of the data is needed and the network does not easily reach the performance of classical methods. To better the classification either additional data of other detectors can be included, to make the identification easier, or customized network layers should be designed. The physical information, which can be implemented through the network design or by preprocessing the data, should increase the ability to classify events. 
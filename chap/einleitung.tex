One objective of the Large Hadron Collider (LHC) at CERN was to be able to produce and identify new particles at very high energies. Several different detectors have been installed that could possibly observe such particles \cite{atlas} \cite{cmscollab}.
The CASTOR calorimeter is a very forward detector, located at the edge of the CMS detector at the LHC. CASTOR can provide information on a number of topics. One of these is the hypothesis of strangelets, a theoretical particle consisting of up, down and strange quarks \cite{strangelet}. Particles with a small number of strange quarks are unstable as the strange quark decays into the lighter quarks via the weak interaction. The \enquote{strange matter hypothesis} states that composite particles with a higher number of strange quarks and roughly the same number of up and down quarks may become stable. As such particles have a very high mass, they could mostly be detected in the very forward region of the CMS detector. Thus, CASTOR was constructed to search for strangelets \cite{strangeletcastor}. To find such events it is crucial to correctly identify characteristic signatures made by known backgrounds and to separate them from possible unknown signals. \newline
Neural networks have many applications nowadays. They are used mostly when very large amounts of data are available and a distinctive pattern can be picked out. Particle physics is therefore a prime field for the application of neural networks \cite{dnnparticle1}\cite{dnnparticle2}\cite{dnnparticle3}. With the help of Monte Carlo simulations features can be learned and detected. In this thesis a neural network is developed to correctly identify and differentiate particle signatures in CASTOR. If this network can correctly classify known events, possible new particle signatures can be found as excess over the background after the classification. A further possibility could be to include the theoretical signature of for example a strangelet and to specifically look for such events in real data. \newline
The event classes given to the network in this thesis consisted of electromagnetic particles, electrons and photons, hadronic particles, such as pions and protons, and background. To further simplify the work of the network, only isolated or single particles are selected as events. With this selection three different data sets have been composed, one containing isolated particles, one single particles and the last being a binary set with only the electromagnetic and hadronic class, omitting the background completely. To train the network Monte Carlo-generated proton proton collisions ad $\sqrt{\mathrm{s}} = 13 \,$TeV generated by PYTHIA8 and EPOS are used. The network is then tested on actual CASTOR events. In the end the application and functionality of the network is also evaluated on low-luminosity data recorded in June 2015. 